\documentclass[12pt, a4paper, leqno]{article}
\usepackage[a4paper, margin=1in]{geometry}
\usepackage{graphicx}
\usepackage{amsmath}
\usepackage[T1]{fontenc}
\usepackage[utf8]{inputenc}
\usepackage{tikz}

\setlength{\parindent}{0pt}
\setlength{\parskip}{1em}


\title{Bølgelikningen oppgave basis}
\author{Henrik Andreas Tømt\thanks{Kadett, CISK}}
\date{september 2025}

\begin{document}

\maketitle

\tableofcontents

\section{Hva er bølgelikningen?}

\subsection{Oppgavetekst oppgave 1}

Bølgelikningen beskriver vibrasjonen av transversale bølger langs en elastist streng.
Strengen har lengde \textit{l}. Se for deg at strengen er i ``ubalanse'' og ved t = 0
slippes og kan vibrere fritt. 

Problemet handler om å bestemme vibrasjonene til strengen. Altså å finne defleksjonen $u(x, t)$ 
på hvilket som helst punkt $x$ på hvilken som helst tid $t > 0$.

For å løse likningen trenger vi noen forhåndsregler: 
\begin{enumerate}
  \item \textit{Massen av strengen per lengde er konstant (``Homogen'') streng. Strengen er pefekt
    elastisk og har ingen resistanse i ``bøying''}
  \item \textit{Spennet / spenningen foråsaket av strekking før fikserinen av endepunktene er såpass
    stor at gravitasjonspvirkningen av strengen er neglisjerbar}
  \item \textit{Strengen beveger seg i små transverse bevegelser langs et vertikalt plan, det er slik
    at hver partikkel på strengen beveger seg kun vertikalt, slik at defleksjonen og helningen på strengen
    langs hvert punkt av strengen forblir små i absolutt verdi}
\end{enumerate}

Disse assumpsjonene er slik at vi kan forvente at løsningen $u(x , t)$ av differensiallikningen av den fysiske 
fysiske ``ikke-idealiserte'' strengen av små homogene masser under sterk spenning. 
\subsection{Utledning}

Å finne løsningen til differensiallikningen må vi vurdere kreftene som virker på en liten del av strengen. 
Siden strengen ikke har noe resistanse mot bøyning, er spenningen tangentiell til avbøyningen til strengen 
for ethvert punkt. La $T_1$ og $T_2$ være spenningene på endepunktene $P$ og $Q$ av den delen. Siden det ikke 
er noe bevegelse horisontal retning, må de horisontale komponentene av spenningen være konstant.

\begin{equation}
  T_1 \cos\alpha = T_2 \cos\beta = T = const.
  \label{eq:konst_T}
\end{equation}

\begin{figure}[h]
  \centering
  \includegraphics[width=0.75\textwidth]{../Figurer/vibrating_string.png}
  \caption{Vibrerende streng} 
\end{figure}

I vertikal retning har vi to krefter, $-T_1 \sin\alpha$ og $T_2 \sin\beta$ av $T_1$ og $T_2$. Vi får 
minustegnet fordi komponentene av $P$ peker nedover. Newtons andre lov sier at resultatet av disse 
kreftene er lik massen $\rho\Delta x$ av porsjonen ganger akkselerasjonen $\partial ^2 u / \partial t^2$;
her $\rho$ er massen av den uavbøyde strengen per enhet lengde, og $\Delta x$ er lengden av porsjonen
av uavbøyd streng.

$T_2 \sin\beta$ er den vertikale komponenten av $T_2$ som peker oppover ($+u$), og $T_1 \sin\alpha$ er den
vertikale komponenten av $T_1$ som peker nedover ($-u$) En kan skrive $T_1 \sin\alpha$ som $T_{1 u} = T_1 \sin\alpha$
og $T_2 \sin\beta$ som $T_{2 u} \sin\beta$ endrer vi masse per lengde, $\rho \Delta x$ med $m$ og akksellerasjon
$\partial^2 u / \partial t^2$ med $a$ kan vi skrive om  slik:

\begin{align*}
  T_{2 u} - T_{1_u} &= ma \\
  T_2 \sin\beta - T_1 \sin\alpha &= \rho \Delta x \frac{\partial ^2 u}{\partial t^2}
\end{align*}

\begin{equation}
  \frac{T_2 \sin\beta}{T_2 \cos\beta} - \frac{T_1 \sin\alpha}{T_1 \cos\alpha} = 
  \tan{\beta} - \tan{\alpha} = \frac{\rho \Delta x}{T} \frac{\partial^2 u}{\partial t^2}
\label{eq:vertikal_komponent}
\end{equation}

$\tan \alpha$ og $\tan \beta$ er avblyningen på kurvene på $x$ og $x + \Delta x$.

\begin{center}
  $\tan \alpha =  {(\frac{\partial u}{\partial x})}_x$ og $\tan \beta = {(\frac{\partial u}{\partial x})}_{x + \Delta x}$.
\end{center}

Ved å bruke partielle deriverte, ettersom $u$ også avhenger av $t$. Deler en (\ref{eq:vertikal_komponent}) på $\Delta x$ får vi

\begin{equation*}
  \frac{1}{\Delta x} \left[ {\left( \frac{\partial u}{\partial x} \right)}_{x + \Delta x}
  - { \left( \frac{\partial u}{\partial x} \right) }_x \right]
  = \frac{\partial \rho}{T} \frac{\partial^2 u}{\partial t^2}
\end{equation*}
\section{Oppgave 2}

tekst her 
\end{document}

