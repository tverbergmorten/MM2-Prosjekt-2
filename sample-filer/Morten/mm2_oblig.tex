% MM2 Oblig1
% ===================================================================================
\documentclass[12pt,a4paper]{article}

% ----------------------- Pakker -----------------------------------------------
\usepackage[T1]{fontenc}                                % Europeisk fontkoding (korrekte æ/ø/å i PDF)
\usepackage[utf8]{inputenc}                             % Inndata-koding (utf8) for pdflatex

% --- Matematikk (unngå redef-kollisjoner) ---
\usepackage{amsmath}                                    % Bedre matemodus, align/multline/gather
\usepackage{amssymb}                                    % Ekstra matematiske symboler
\usepackage{amsthm}                                     % Teorem-, lemma- og bevis-miljøer
\usepackage{mathtools}                                  % Utvidelser til amsmath (f.eks. \coloneqq)
\usepackage{bm}                                         % Fete greske bokstaver (\bm)

% --- Skrifter, mikrotypografi, bilder ---
\usepackage{newtxtext}                                  % Times-lignende brødtekst
\usepackage{microtype}                                  % Mikrosperring/kerning for penere avsnitt
\usepackage{graphicx}                                   % Inkludere bilder (\includegraphics)

% --- Struktur og lister ---
\usepackage{titlesec}                                   % Tilpasse seksjonsoverskrifter
\usepackage{enumitem}                                   % Kontroll på listeinnrykk/spacing

% --- Lenker og referanser ---
\usepackage[unicode]{hyperref}                                   % Klikkbare lenker/refs i PDF, metadata

% --- Sideoppsett og enheter ---
\usepackage[a4paper,margin=2.5cm]{geometry}             % Sidestørrelse og marger
\usepackage{siunitx}                                    % SI-enheter og tallformatering (\SI, \num)
\usepackage{setspace}                                  % Justere linjeavstand (\setstretch)

% --- Innholdsfortegnelse og fysikknotasjon ---
\usepackage{tocloft}                                    % Finjustere innholdsfortegnelsen
\usepackage{physics}                                    % Vanlige fysikkmakroer (\dv, \qty, …)

% --- Språk ---
\usepackage[main=norsk,english]{babel}                  % Orddeling/språk (norsk hovedspråk)
\addto\captionsnorsk{%
  \renewcommand{\contentsname}{Innholdsfortegnelse}%
}

% --- Redegjør for matte i overskrifter ---
\pdfstringdefDisableCommands{%
  \def\frac#1#2{#1/#2}%
  \def\left( {(} \def\right) {)}%
}


% ----------------------- Formatering ---------------------------------------
% --- Avsnitt ---
\setlength{\parindent}{0pt}                                                               % Ingen innrykk på første linje i avsnitt
\setlength{\parskip}{0.6\baselineskip plus 2pt minus 1pt}                                 % Dynamisk luft mellom avsnitt (kan strekkes/krympes)

% --- Linjeavstand (litt løsere) --- 
\setstretch{1.10}                                                                         % Litt løsere linjeavstand (typisk 1.10–1.15)

% --- Luft rundt display-matte (equation/align/...) ---
\makeatletter                                                                             
\g@addto@macro\normalsize{%                                                               % Justér standard-verdier for normalsize
  \setlength\abovedisplayskip{-0.3\baselineskip plus 1pt minus 1pt}                        % Luft over display-likninger
  \setlength\belowdisplayskip{1.3\baselineskip plus 1pt minus 1pt}                        % Luft under display-likninger
  \setlength\abovedisplayshortskip{-0.3\baselineskip plus 1pt minus 1pt}                   % Når linja er “kort” før display
  \setlength\belowdisplayshortskip{1.3\baselineskip plus 1pt minus 1pt}                   % Tilsvarende under
}
\makeatother                                                                             

% --- Seksjonsavstand (med titlesec) ---
\titlespacing*{\section}
  {0pt}{2\baselineskip plus .4\baselineskip minus .2\baselineskip}{0.2\baselineskip}    % Venstreinnrykk, før- og etter-avstand
\titlespacing*{\subsection}
  {0pt}{1.1\baselineskip plus .3\baselineskip minus .2\baselineskip}{0\baselineskip}    % Litt strammere for subsection

% --- Lister (med enumitem) ---
\setlist{itemsep=0.4\baselineskip plus 2pt minus 1pt, topsep=0.5\baselineskip}   

% --- Innholdsfortegnelse (med tocloft) ---
\setlength{\cftbeforesecskip}{0.25\baselineskip}                                          % Luft før hver section-linje i ToC
\setlength{\cftbeforesubsecskip}{-0.15\baselineskip}                                      % Luft før hver subsection-linje i ToC

%------------------------Pakke-konflikt-----------------------------
% siunitx vs physics: bruk siunitx sin \qty
\AtBeginDocument{\RenewCommandCopy\qty\SI}
\ExplSyntaxOn
\msg_redirect_name:nnn { siunitx } { physics-pkg } { none }
\ExplSyntaxOff

\emergencystretch=1em

%------------------------------Metadata-kommandoer------------------------------
\newcommand{\institution}{FHS/Cyberingeniørskolen}
\newcommand{\coursecode}{ING2501 \; Matematiske Metoder 2}
\newcommand{\studentname}{Morten K. Tverberg}
\newcommand{\class}{VING 78}
\newcommand{\submissiondate}{\today}
\newcommand{\documenttitle}{Obligatorisk Oppgave}
\newcommand{\documentsubtitle}{Taylorrekker}


%----------------------- Stil / Pre-amble --------------------------------------
\pagestyle{plain}                           

\hypersetup{
  colorlinks=true,
  linkcolor=black,
  urlcolor=black,
  citecolor=black,
  pdfauthor={\studentname},
  pdftitle={\documenttitle}
}

\titleformat{\section}{\large\bfseries}{\thesection.}{0.5em}{}
\titleformat{\subsection}{\normalsize\bfseries}{\thesubsection}{0.5em}{}
\titleformat{\subsubsection}{\normalsize\itshape}{\thesubsubsection}{0.5em}{}


%----------------------- Forside --------------------------------------
\newcommand{\makedoctitle}{%
  \thispagestyle{empty}
  \begin{minipage}{0.15\textwidth}
      \includegraphics[height=3cm]{cisk.png}
  \end{minipage}%
  \begin{minipage}{0.5\textwidth}
      \vspace{10ex}
      {\bfseries \institution}\\
      {\itshape \coursecode}
  \end{minipage}% 
  \begin{minipage}{0.3\textwidth}
      \vspace{13ex}
      \raggedleft
      \today 
  \end{minipage}\\[0.5ex]
  {\rule{\textwidth}{2pt}} \\ \vspace{3ex}

  \begin{center}
      {\Huge \bfseries \documenttitle} \\[1ex]
      {\LARGE \bfseries \documentsubtitle} \\[2ex]
      {\normalsize \studentname} \\[0.5ex]
      {\normalsize \class} \\[1ex]
  \end{center}

\clearpage
}

%----------------------- Dokument -----------------------
\begin{document}

\makedoctitle
\pagenumbering{arabic}
\setcounter{page}{1}

\tableofcontents
\clearpage

% -------------------------------------------------- Oppgave 1 --------------------------------------------------
\section{Taylorpolynom}
\subsection{Bestem $P_{2,0}(x)$}
Vi har funksjonen $S(x)=e^x$. 
Ved hjelp av andre ordens taylorpolynom kan vi gjøre om $S(x)$ til et polynom rundt $x=0$. 
Dette gir oss:

\begin{equation*}
  P_{2,0}(x) = S(0) + S'(0)x + \frac{S''(0)}{2}x^2
\end{equation*} \vspace{1ex}


Siden den deriverte av funksjonen $S(x)$ er lik funksjonen selv, får vi at:

\begin{equation*}
  P_{2,0}(x) = 1 + x + \frac{1}{2}x^2
\end{equation*} \vspace{1ex}


\subsection{Finn en tilnærmet verdi for $e^{0.2}$}
For å finne en tilnærmet verdi for $S(0.2) = e^{0.2}$ kan vi bruke taylorpolynomet vi fant i $1.1$. 
Vi setter inn $x=0.2$ i polynomet og får denne likningen:

\begin{equation*}
  P_{2,0}(0.2) = 1 + 0.2 + \frac{1}{2}(0.2)^2 \approx 1.22
\end{equation*} \vspace{3ex}


% -------------------------------------------------- Oppgave 2 --------------------------------------------------
\section{Taylorpolynom fra differensialligning}
\subsection{Bestem $P_{2,0}(x)$}
Vi har funksjonen $y(x)\cdot(y'(x)+x+1)=6$ med startbetingelsen $y(0)=3$. 
Vi ønsker å finne et andre ordens taylorpolynom rundt $x=0$. Dette gir oss:

\begin{equation*}
  P_{2,0}(x) = y(0) + y'(0)x + \frac{y''(0)}{2}x^2
\end{equation*}

\clearpage

Her kan vi sette inn startbetingelsen i første ledd. 
For å få andre ledd må vi stokke om på funksjonen slik at vi får:

\begin{equation*}
  y'(x) = \frac{6}{y(x)} - x - 1
\end{equation*}


Ved å sette inn $x=0$ i $y'(x)$ får vi:

\begin{equation*}
  y'(0) = \frac{6}{y(0)} - 0 - 1 = \frac{6}{3} - 1 = 1
\end{equation*}


Videre kan vi finne den tredje funksjonsverdien ved å derivere $y'(x)$:

\begin{equation*}
  y''(x) = -\frac{-y'(x) \cdot (y'(x)+x+1)}{(y(x))} - 1
\end{equation*}


Hvis vi setter inn $x=0$ i $y''(x)$ får vi:

\begin{equation*}
  y''(0) = -\frac{-y'(0) \cdot (y'(0)+0+1)}{(y(0))} - 1 
  = -\frac{-1 \cdot (1+1)}{3} - 1 
  = \frac{2}{3} - 1 
  = -\frac{1}{3}
\end{equation*}



Dette kan videre settes inn i polynomet, for å få:

\begin{equation*}
  P_{2,0}(x) = 3 + x - \frac{1}{3}x^2
\end{equation*}


\subsection{Finn en tilnærmet verdi for $y(\frac{1}{10})$}
For å finne en tilnærmet verdi av $y(\frac{1}{10})$ kan vi sette inn $x=\frac{1}{10}$ i polynomet vi fant i $2.1$. 
Dette gir oss:

\begin{equation*}
  P_{2,0} \left( \frac{1}{10} \right) 
  = 3 + \frac{1}{10} - \frac{1}{3} \cdot \left( \frac{1}{10} \right)^2 
  = \frac{929}{300}
\end{equation*}

\clearpage


% -------------------------------------------------- Oppgave 3 -------------------------------------------------- 
\section{Maclaurinrekke bevis + integralregning}
\subsection{Bevis for sammenheng mellom macalurinrekke og potensrekke}
Vi skal i denne oppgaven bevise at:

\begin{equation*}
  \frac{x-\sin{x}}{x^3} \hspace{2ex} 
  = \hspace{2ex} \sum_{n=0}^{\infty} \hspace{0.5ex} (-1)^n \hspace{0.5ex} \frac{x^{2n}}{(2n+3)!}
\end{equation*}


Vi begynner med å finne en tilsvarende maclaurinrekke for $\sin{x}$, 
som gjøres ved å bruke definisjonen av maclaurinrekke:

\begin{equation*}
  \sin{x} = x-\frac{x^3}{3!}+\frac{x^5}{5!}-\frac{x^7}{7!}+\frac{x^9}{9!} -
  \hspace{1ex}\dots
\end{equation*}


Hvis vi legger til $x-$ og $\frac{1}{x^3}$ delen av stykket, får vi følgende:

\begin{equation*}
  \frac{x-\sin{x}}{x^3} = 
  \frac{x - x-\frac{x^3}{3!}+\frac{x^5}{5!}-\frac{x^7}{7!}+\frac{x^9}{9!}-\hspace{0.5ex}\dots}{x^3}
   = -\frac{1}{3!}+\frac{x^2}{5!}-\frac{x^4}{7!}+\frac{x^6}{9!}-\hspace{1ex}\dots
\end{equation*}


Her ser vi et tydelig mønster, 
hvor vi har en $x$ som øker partallsvis med vekslende fortegn, og en fakultet som øker oddetallsvis. 
Dette kan vi skrive som følgende sum:

\begin{equation*}
  \frac{x-\sin{x}}{x^3} = \sum_{n=0}^{\infty} (-1)^n \frac{x^{2n}}{(2n+3)!}
\end{equation*}
\qed

\subsection{Utregning av integral}
Neste oppgave går ut på å regne integralet:

\begin{equation*}
  \int_0^1 \frac{x-\sin{x}}{x^3} \dd{x}
\end{equation*}


For å gjøre integralet lettere å regne ut, 
kan vi ved å bruke potensrekken vi fant i $3.1$, 
skrive om integralet til:

\begin{equation*}
  \sum_{n=0}^{\infty} (-1)^n \frac{1}{(2n+3)!} 
  \hspace{1ex} \int_0^1 x^{2n} \dd{x}
\end{equation*}


Selve integralet kan vi regne ut som:

\begin{equation*}
  \int_0^1 x^{2n} \dd{x} = \left[ \frac{x^{2n+1}}{2n+1} \right]_0^1 = \frac{1}{2n+1}
\end{equation*}


Dette kan vi sette inn i summen, som da blir:

\begin{equation*}
  \sum_{n=0}^{\infty} (-1)^n \frac{1}{(2n+3)!} \cdot \frac{1}{2n+1}
  = \sum_{n=0}^{\infty} (-1)^n \frac{1}{(2n+1)(2n+3)!}
\end{equation*}


For å avgjøre om summen konvergerer, kan vi regne de første tre leddene individuellt.

\begin{equation*}
  \begin{split}
  & n=0: \hspace{4.2ex} \frac{1}{3!} \hspace{2ex} \approx \hspace{1ex} 0.16667 \\[1ex]
  & n=1: \hspace{1ex} -\frac{1}{5 \cdot 5!} \hspace{1ex} \approx \hspace{1ex} -0.00167 \\[1ex]
  & n=2: \hspace{2.8ex} \frac{1}{7 \cdot 7!} \hspace{1ex} \approx \hspace{1ex} 0.0000283
  \end{split}
\end{equation*}


Vi ser her at for økende $n$-verdi vil likningene i summen gå mot 0. 
Dette betyr at for å finne en tilnærmet verdi av summen fra $0 \rightarrow \infty$, 
kan vi bruke de fem første leddene eksempelvis:

\begin{equation*}
  \sum_{n=0}^{4} (-1)^n \frac{1}{(2n+1)(2n+3)!} 
  =  \frac{1}{3!} - \frac{1}{5 \cdot 5!} + \frac{1}{7 \cdot 7!} - \frac{1}{9 \cdot 9!} + \frac{1}{11 \cdot 11!}
  \approx 0.1639
\end{equation*}


% -------------------------------------------------- Oppgave 4 --------------------------------------------------
\section{Potensrekke}
\subsection{Konvergens av potensrekken}
I denne oppgaven skal vi finne konvergens av rekken:

\begin{equation*}
  a_n = \sum_{n=0}^{\infty} \hspace{0.5ex} (-1)^n \hspace{0.5ex} \frac{x^{n+1}}{n!}
\end{equation*}


For å finne konvergensen kan vi bruke formelen:

\begin{equation*}
  \lim_{n \to \infty} \left| \frac{a_{n+1}}{a_n} \right|
\end{equation*}


hvor

\begin{equation*}
  a_{n+1} = \sum_{n=0}^{\infty} \hspace{0.5ex} (-1)^{n+1} 
  \hspace{0.5ex} \frac{x^{n+2}}{(n+1)!}
\end{equation*}


Hvis vi legger formelene inn i konvergensformelen, får vi:

\begin{equation*}
  \lim_{n \to \infty} \left| \frac{(-1)^{n+1} \cdot \frac{x^{n+2}}{(n+1)!}}{(-1)^n \cdot \frac{x^{n+1}}{n!}} \right|
  = \lim_{n \to \infty} \left| -\frac{x^{n+2} \cdot n!}{(n+1)! \cdot x^{n+1}} \right|
  = \lim_{n \to \infty} \left| -\frac{x}{n+1} \right|
  = 0
\end{equation*}


Dette betyr at rekken konvergerer for alle $x \in \mathbb{R}$.


\subsection{Finne summen av potensrekken}
For å finne summen av rekken kan vi se på $e^x$ sin maclaurinrekke:

\begin{equation*}
  e^x = \sum_{n=0}^{\infty} \hspace{0.5ex} \frac{x^n}{n!} = 1 + x + \frac{x^2}{2!} + \frac{x^3}{3!} + \hspace{1ex}\dots
\end{equation*}


Hvis vi ganger begge sider med $x$, får vi:

\begin{equation*}
  \begin{split}
    xe^x = \sum_{n=0}^{\infty} \hspace{0.5ex} \frac{x^{n+1}}{n!} \\
    {\scriptstyle (x^{n+1} = \hspace{0.5ex} x^n x)}
  \end{split}
\end{equation*}


Stykket begynner allerede å ligne på vårt stykke, 
for å få vekslende fortegn kan vi gange med $-1$:

\begin{equation*}
  \begin{split}
  -xe^x = \sum_{n=0}^{\infty} \hspace{0.5ex} (-1)^n \hspace{0.5ex} \frac{x^{n+1}}{n!} \\
  {\scriptstyle ((-x)^n = \hspace{0.5ex} (-1)^n x^n)}
  \end{split}
\end{equation*}

Dette betyr at potensrekken vi har er lik $-xe^x$.

\end{document}