\subsection{Bakgrunn og målsetting}
I dette prosjektet analyseres et lydklipp fra sangen “Seven Nation Army” av The White Stripes. 
Sangen er valgt fordi den har et tydelig og repeterende bassmotiv med lave frekvenser, som gjør den 
spesielt egnet for analyse ved hjelp av Fourier-transformasjoner. 
De stabile tonene i basslinjen gjør det mulig å undersøke frekvenspresisjon, mens 
trommerytmen gir naturlige tidsvariasjoner som er interessante for Short-Time Fourier Transform (STFT).



\subsection{Problemstilling}
Hovedproblemstillingen som denne oppgaven søker å besvare er:
\textit{Hvordan kan Fast Fourier-transformasjonen (FFT) benyttes til å analysere et digitalt lydsignal for å 
identifisere de mest dominerende frekvenskomponentene i en sang?}.



\subsection{Hvordan kunstig inteligens er brukt i arbeidet med rapporten}
\dots

